\section{Introduction}
Due to different retinal degenerative diseases, many people 
lose their vision. Such loss significantly reduces their abilities to 
successfully participate in everyday life and the number of tasks they
are able to accomplish. To facilitate their life and provide a way to 
handle new problems, implants (also called \textit{bionic vision implants}) 
could be installed providing an ersatz-sight that is far from reality but still
uses the same way to provide information to the human brain.  

Despite continuous research and constant improvements in this area, 
modern approved bionic implants still provide very low resolution. 
This problem heavily restrains implant owners and reduces the number 
of tasks that could be reliably solved relying on visual information. 
One of these tasks is objects identification and visual differentiation. 
With such low resolution (up to 40 pixels for each axis), it is hard to 
visually disjunct near-located objects and to locate needed objects in indoor scenes.

In this work, we aim to provide a method to improve visual information available 
to low-resolution implant owners to allow them more easily understand the number of 
objects on the scene and visually separate them. We make the next contributions:
\begin{itemize}
    \item We adjust and modify state-of-the-art computer vision algorithms based on neural networks 
    and graph theory to modify visual observations and introduce more information to implant owners
    \item We evaluate our solution using open-source implementations of computational models of bionic 
    vision and hold an unbiased comparison of our solution with plain grayscale image data.   
\end{itemize}